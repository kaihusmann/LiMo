\documentclass{article}
\usepackage{amsmath}
\begin{document}

\textbf{Sheet 6}\\
Sabastian \& Kai\\
(Die Zusatzaufgabe Sheet 4 letzter Woche könnte in der Klausur dran kommen.)\\

\textbf{1a}\\
See the markdown file on LiMo Homepage. The example ist only for expanation of Bayer inference. The procedure is often not possible because the posterior is often unknown/not calculatable.\\


\textbf{2a}\\
\begin{align*}
P(\tau) &= \frac{\delta^{\alpha}}{\Gamma(\alpha)} \tau^{\alpha-1} \exp(-\delta \tau)\\
\text{where} \quad \Gamma(\alpha) &= \int^\infty_{0} x^{\alpha-1} \exp(-x) dx \\
\thinspace \\
P(\tau) &= \int_0^\infty \frac{\delta^{\alpha}}{\Gamma(\alpha)} \tau^{\alpha-1} \exp(-\delta \tau) d \tau\\
&= \frac{1}{\Gamma(\alpha)} \int_0^\infty \delta^{\alpha} \tau^{\alpha-1} \exp(-\delta \tau) d \tau\\
&\propto  \int_0^\infty \delta^{\alpha} \tau^{\alpha-1} \exp(-\delta \tau) d \tau\\
&=  \int_0^\infty (\delta \tau)^{\alpha} \tau^{-1} \exp(-\delta \tau) d \tau && \mu = \tau \delta \text{;}\quad \frac{d \tau}{d \mu}=\frac{1}{\delta}\\
&= \int_0^\infty \mu^\alpha \left( \frac{\mu}{\delta} \right)^{-1} exp(-\mu) \frac{1}{\delta} d\mu\\
&= \int_0^\infty \mu^{\alpha} \mu^{-1} \exp(-\mu) d\mu\\
&= \int_0^\infty \mu^{\alpha - 1} \exp(-\mu) d\mu && \text{Dies ist die Gamma Funktion.}\\
& \text{Mit dem zuvor ausgelassenem Term ergibt sich also:}\\
&=\frac{1}{\Gamma(\alpha)}\Gamma(\alpha)\\
&= 1
\end{align*}

Funktioniert, da $\tau \ge 0$, $\delta > 0$, $\alpha > 0$ and $\Gamma(\alpha)>0$ $\rightarrow$ $p(\tau) \ge 0$\\
\newpage

\textbf{2b}\\
\begin{align*}
E(\tau) &= \int_0^\infty p(\tau) d \tau \\
&= \int_0^\infty \frac{\delta^{\alpha}}{\Gamma(\alpha)} \tau^{\alpha-1} \exp(- \delta \tau) d \tau \\
&= \frac{1}{\Gamma(\alpha)} \int_0^\infty (\delta \tau)^{\alpha} \exp(- \delta \tau) d \tau && \mu=\tau \delta\\
&= \frac{1}{\Gamma(\alpha)} \int_0^\infty (\mu)^{\alpha} \exp(- \mu) \frac{1}{\delta} d \mu \\        
&= \frac{1}{\Gamma(\alpha)} \frac{1}{\delta} \int_0^\infty (\mu)^{\alpha} \exp(- \mu) d \mu && \text{dies ist wieder so aehnlich wie die Gamma Funktion (fehlt ein -)}\\
&= \frac{1}{\Gamma(\alpha)} \frac{1}{\delta} \Gamma(\alpha+1)\\
&= \frac{\alpha}{\delta}
\end{align*}

\begin{align*}
Var(\tau)&= \int_0^\infty \tau^2 \frac{\delta^\alpha}{\Gamma(\alpha)} \tau^{\alpha-1} exp(-\delta \tau) - (\frac{\alpha}{\delta})^2 && \text{Verschiebungssatz} \\
&= \frac{\delta^\alpha}{\Gamma(\alpha)} \int_0^\infty \tau^{\alpha+1} exp(-\delta \tau) - (\frac{\alpha}{\delta})^2 && \text{Schritt unklar, Gamma Funktion??}\\
&= \frac{\delta^\alpha}{\Gamma(\alpha)} \delta^{-(\alpha+2)}\Gamma(\alpha+2)-(\frac{\alpha}{\delta})^2\\
&= \frac{\delta^{\alpha-2}}{\Gamma(\alpha)} \Gamma(\alpha+2)-(\frac{\alpha^2}{\delta^2}) & \Gamma(\alpha+1) &= \alpha \Gamma(\alpha)\text{;} \quad \Gamma(\alpha+2) = (\alpha+1)\alpha \Gamma(\alpha)\\
&=\frac{(\alpha+1)\alpha}{\delta^2}-\frac{\alpha^2}{\gamma^2}\\
&=\frac{\alpha}{\delta^2}\\
\end{align*}

\textbf{3}
\begin{align*}
y &= X\beta + \epsilon \\
\epsilon &\sim N_p(0, \sigma^2 I) \\
\beta &= (\beta_1, \hdots, \beta_p)' \\
\sigma^2 \sim IG(a,b)
\end{align*}

a) posterior disribution of $\sigma^2$
\begin{align*}
p(\sigma^2 | y,X,\beta) &= f(y|X, \beta, \sigma^2) p(\sigma^2) \\ 
f(y|X, \beta, \sigma^2) &= \frac{1}{(2 \pi)^{p/2} det(I \sigma^2)^{1/2}} 
\exp(-\frac{1}{2}(y-X\beta)' (I \sigma^2)^{-1} (y-X\beta))  \\
&\propto \frac{1}{ \sqrt{ (\sigma^2)^n det(I }} 
\exp(-\frac{1}{2}(y-X\beta)' (I \sigma^2)^{-1} (y-X\beta))  \\
&\propto \frac{1}{ (\sigma^2)^{n/2} } 
\exp(-\frac{1}{2}(y-X\beta)' (I \sigma^2)^{-1} (y-X\beta))  
\end{align*}

Prior
\begin{align*}
p(\sigma^2)  &= \frac{b^a}{\Gamma(a)} (\sigma^2)^{-(a+1)} \exp(-b/\sigma^2)  \\
&\propto  (\sigma^2)^{-(a+1)} \exp(-b/\sigma^2)  
\end{align*}

\begin{align*}
p(\sigma^2 | y,X,\beta) &\propto \frac{1}{ (\sigma^2)^{n/2} } 
\exp(-\frac{1}{2}(y-X\beta)' (I \sigma^2)^{-1} (y-X\beta))  (\sigma^2)^{-(a+1)} \exp(-b/\sigma^2)  \\
&= (\sigma^2)^{-(a+1+n/2)} \exp(-\frac{1}{\sigma^2}(\frac{1}{2}(y-X\beta)' (y-X\beta) + b)) 
\end{align*}

\begin{align*}
\sigma^2 | y,X,\beta \sim IG(a+n/2, \frac{1}{2}(y-X\beta)' (y-X\beta) + b)
\end{align*}

\textbf{b}
flat prior = non informative prior
we get flat priors if the IG distribution is constant. Thus the prior is flat for the values

$b=0, a=-1 p(\sigma^2) = const$

\begin{align*}
p(\sigma^2)  &\propto  (\sigma^2)^{-(a+1)} \exp(-b/\sigma^2)  
\end{align*}

\textbf{c}
\begin{align*}
\hat{\sigma}^2 = \frac{b_{post}}{a_{post}+1} = \frac{\frac{1}{2}(y-X\beta)' (y-X\beta) + b}{a+n/2+1}
\end{align*}

fuer $b=0, a=-1$ 
\begin{align*}
\hat{\sigma}^2 = \frac{\frac{1}{2}(y-X\beta)' (y-X\beta)}{n/2} = \frac{1}{n} (y-X\beta)' (y-X\beta) = \hat{\sigma}_{ML}^2
\end{align*}

\textbf{4}
\begin{align*}
y \sim Po(\lambda) \\
f(y|\lambda) = \frac{\lambda^y_i}{y_i!}\exp(-\lambda)
\end{align*}

Likelihood
\begin{align*}
L(y|\lambda) = \prod_i f(y_i|\lambda) = \prod_i \frac{\lambda^y_i}{y_i!}\exp(-\lambda) \\
=  \frac{1}{\prod_i y_i!} \lambda^{\sum_i y_i} \exp(-n \lambda)
\end{align*}

Prior (Gamma Verteilung)
\begin{align*}
p(\lambda)  &= \frac{b^a}{\Gamma(a)} (\lambda^2)^{-(a+1)} \exp(-b/\lambda)  \\
&\propto  (\sigma^2)^{a-1} \exp(-b\sigma^2)  
\end{align*}

Posterior
\begin{align*}
p(\lambda | y) &\propto f(y|\lambda) p(\lambda)  \\
&\propto \lambda{\sum_i y_i} \exp(-n \lambda) \lambda^{a-1} \exp(-b \lambda) \\
&= \lambda{ a + \sum_i y_i -1 } \exp(-(b+n)\lambda)
\end{align*}

\begin{align*}
\lambda | y \sim Ga(a + \sum_i y_i, b+n)
\end{align*}

Konjugierte Verteilung, da der Kern der Post.-Verteilung wieder einer Gamma Vertielung entspricht.\\
$p(\lambda | y)$ hat die Form einer Gamma Verteilung mit $a=a+\sum y_i4$ und $b=b+n$, also $\lambda(y  \sim Ga(a+\sum y_i, b+n) )$\\

\textbf{5} R. There was a mistake in an early version of the sheet in b.\\
\newpage
\textbf{6} see page 284\\

\begin{align*}
u(\tau | y) \rightarrow y(\tau, \beta |y) &= cP(\tau, \beta) L(\tau, \beta| y)\\
P(\tau, \beta) &= P_1 \tau|\beta) P_2(\tau)\\
\text{where} \quad \beta|\tau \sim N_{p+1}(\psi \tau^{-1} V) \quad \text{and} \quad
\tau \sim G(\alpha, \sigma)\\
\end{align*}
\begin{align*}
L(y|\beta, \tau) &= \prod_i \frac{1}{(2 \pi)^{p/2} det(\tau^{-1}I)} \frac{1}{2} \exp(-\frac{1}{2}(y-X\beta)' (\tau I) (y-X\beta))\\
&\propto (\tau^n)^{1/2} \exp(-\frac{\tau}{2}(y-X\mu)'  (y-X\beta))\\
\end{align*}
\begin{align*}
P_2(\tau) &= \frac{\sigma^\alpha}{\Gamma(\alpha)} \tau^{(\alpha-1)} \exp(-\sigma \tau)\\ 
&\propto  \tau^{(\alpha-1)} \exp(-\sigma \tau)\\
\end{align*}
\begin{align*}
P_1(\beta|\tau) &= \frac{1}{(2 \pi)^{\frac{k+1}{2}} det(\Sigma)^{1/2}} 
\exp(\tau (\beta-\psi)' V^{-1} \frac{(\beta-\psi)}{2})\\
&\propto \tau^{\frac{k+1}{2}}\exp(-\tau (\beta-\psi)' V^{-1} \frac{(\beta-\psi)}{2})\\
\end{align*}
\begin{align*}
P(\beta, \tau) &\propto P_1 P_2\\
&\propto \tau^{\frac{k+1}{2}} + \frac{2(\alpha-1)}{2} \exp(\frac{-\tau(\beta-\psi)'V^{-1}(\beta-\psi)}{2}-\frac{2 \sigma \tau}{2})\\
\end{align*}
\begin{align*}
y(\tau,\beta|y)&=P(\beta,\tau)L(\tau,\beta|y)\\
&\propto \tau^{\frac{k+1+\alpha+n}{2}}exp(-\frac{\tau}{2}[\beta \psi'V^{-1}(\beta-\psi)+2 \sigma + (y-X\beta)'(y-X\beta)])\\
\end{align*}
$\rightarrow$ post. Verteilung $G(\beta, \tau|y)$\\

$c_1$ [in (11.9) auf Seite 281] $=\frac{1}{2 \pi^{\frac{k+1}{2}}}$\\

$c_2$ [in(11.10)] $=c_1 \frac{1}{2 \pi^{\frac{n}{2}}}$\\
\begin{align*}
\delta_{*} \text{[in(11.14)]} &=2\delta\\
\alpha_{**} \text{[in(11.14)]} &=2\alpha-2+n\\
\delta_{**} \text{[in(11.14)]} &=-\psi'_*V^{-1}_*+\psi V^{-1} \psi \dots \\
\end{align*}
Die Konstandten die zuvor weggelassen wurden muessen jetzt multiplikativ hinzugefügt werden, da das Integral der Dicht ansonsten nicht 1 ergeben wuerde.\\
$c_5$ [in(11.14)] $=c_2 |V_*|^{1/2} *(2\pi)^{\frac{k+1}{2}}$\\

\textbf{5b} Change of Variable technique\\
\begin{align*}
(\sigma^2|y) &= v(\tau|y)|\frac{d \tau}{d \sigma^2}|\\
&= v(\frac{1}{\sigma^2}|y)*\frac{1}{\sigma^4}\\
&=c_5(\frac{1}{\sigma^2 })^{\frac{\alpha+n}{2}-1}\frac{1}{\sigma^4}-exp\left(-\left[\left(\frac{-\psi'_* V^{-1}_* \psi_* + \psi' V^{-1} \psi + y'y+2 \psi}{2}\right) \right] \frac{1}{\sigma^2} \right)\\
&=c_5(\sigma^2)^{-\left(\frac{\alpha+n}{2}-1\right)-2}exp\left(\dots\right)\\
&=c_5(\sigma^2)^{-\frac{\alpha+n}{2}+1-2}exp\left(\dots\right)\\
&=c_5(\sigma^2)^{-\frac{\alpha+n}{2-1}}exp\left(\dots\right)\\
\end{align*}

\end{document}